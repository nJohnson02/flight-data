\documentclass[conference]{IEEEtran}
\usepackage[utf8]{inputenc}
\usepackage{amsmath,amssymb,amsfonts}
\usepackage{graphicx}
\usepackage{cite}
\usepackage{hyperref}

\title{Early Classification of Landing Quality Using Time-Series Flight Data and Gradient Boosting}

\author{\IEEEauthorblockN{Nathan Johnson}
\IEEEauthorblockA{\textit{Embry-Riddle Aeronautical University} \\
Prescott, Arizona, USA \\
nathan.johnson@erau.edu}}

\begin{document}

\maketitle

\begin{abstract}
This paper presents a machine learning approach for early classification of aircraft landings as either good or bad, using time-series data from onboard flight logs. We explore feature extraction techniques from GPS altitude, derived descent rate, and indicated airspeed, and train a gradient boosting classifier using a curated set of manually labeled flight approaches. The model demonstrates strong accuracy when full approach data is used, and is further evaluated under constraints where only partial approach data is available. Results suggest early indicators of landing quality exist, though bad landings become more apparent closer to touchdown.
\end{abstract}

\begin{IEEEkeywords}
Aviation safety, machine learning, gradient boosting, landing prediction, time-series classification
\end{IEEEkeywords}

\section{Introduction}
Safe and stable aircraft landings are a critical component of aviation safety. This work investigates whether the quality of a landing can be predicted using a time-series of sensor values recorded during approach. Specifically, we train a machine learning model to classify landings as ``good'' or ``bad'' using data collected from general aviation aircraft.

\section{Background}
As a pilot, I have insights into parameters influencing landing quality. Previous work, such as the 2021 study on commercial flight landings \cite{chen2016xgboost}, provides valuable methodologies that can be adapted for general aviation. While that study focused on commercial aircraft, many principles, such as analyzing descent profiles, airspeed trends, and flare techniques are applicable to this project.

\section{Methodology}

\subsection{Data Collection}
Flight logs were obtained from a series of general aviation approaches at KPRC airport. Each log contains a CSV file with time-series data including GPS altitude (AltGPS), vertical speed (VSpd), and indicated airspeed (IAS). The final 60 seconds of each approach were extracted, and each approach was manually labeled as good or bad using custom visual inspection tools.

\subsection{Preprocessing and Feature Engineering}
For each approach, the following features were computed:
\begin{itemize}
    \item AltGPS (raw altitude samples)
    \item Altitude Rate: first derivative of AltGPS (ft/min)
    \item IAS (when applicable)
\end{itemize}
The result is a 1D feature vector of fixed size per approach.

\subsection{Model Training}
We used \texttt{XGBClassifier} from the \texttt{xgboost} library \cite{chen2016xgboost}. Hyperparameters included 100 estimators and max depth of 5. Models were evaluated using 5-fold cross-validation. Performance was also measured on an 80/20 train/test split for reporting confusion matrices and classification scores.

\section{Results}

Using all 60 seconds of data:
\begin{itemize}
    \item Accuracy: 97.5\%
    \item Bad landing recall: 94\%
    \item Good landing precision: 99\%
\end{itemize}

Using only the first 40 seconds of data:
\begin{itemize}
    \item Accuracy: 83.5\%
    \item Bad landing recall: 23\%
    \item Good landing recall: 95\%
\end{itemize}

Adding IAS had minimal effect on early classification.

\section{Discussion}
The model is highly effective at identifying good landings early in the approach. However, bad landings are much harder to detect until the final seconds, which is consistent with human pilot judgment. Feature engineering beyond basic altitude metrics may be necessary to improve early recall of bad landings.

\section{Conclusion}
This work demonstrates that landing quality prediction is feasible using time-series data from flight logs. A gradient boosting classifier trained on altitude-derived features achieves strong accuracy when the full approach is used. Predicting landing quality early remains challenging, especially for bad landings, and presents opportunities for future feature development.

\bibliographystyle{IEEEtran}
\bibliography{references}

\end{document}
